\begin{tikzpicture}[-latex, auto, node distance =4 cm and 5cm, on grid , semithick, scale = 1.0,
            state/.style ={ circle ,top color =white , bottom color = processblue!20 ,
            draw,processblue , text=blue , minimum width =1 cm, scale = 1.0}]
        
        %\node [estilo del nodo] (nombre del nodo) [posicion relativa]/at (posicion,absoluta) {};
        \node[state] (n1) at (-5, 0) {$x_1$};
        \node[state] (n2) at (-2.5, 2.5) {$x_2$};
        \node[state] (n3) at (-2.5, -2.5) {$x_3$};
        \node[state] (n4) at (2.5, -2.5) {$x_4$};
        \node[state] (n5) at (2.5, 2.5) {$x_5$};
        \node[state] (n6) at (5, 0) {$x_6$};
        

        \path (n1) edge [bend right = -25] node[above = 0.15 cm] {$1$} (n2);
        \path (n1) edge [bend right = 25] node[below = 0.0 cm] {$2$} (n3);
        \path (n2) edge  node[above = 0.15 cm] {$31$} (n5);
        \path (n3) edge  node[above = 0.15 cm] {$12$} (n4);
        \path (n5) edge  [bend right = -25] node[above = 0.15 cm] {$4$} (n6);
        \path (n4) edge  [bend right = 25] node[below = 0.15 cm] {$2$} (n6);
        \path (n1) edge  node[above = 0.15 cm] {$30$} (n5);
        \path (n2) edge  [bend right = 25] node[above = 0.15 cm] {$15$} (n4);
        %si queremos un loop
        %\path (A) edge [loop left] node[left] {$1/4$} (A);
    \end{tikzpicture}