\begin{figure}[h]

    \begin{tikzpicture}[-latex, auto, node distance =4 cm and 5cm, on grid , semithick, scale = 1.0,
               state/.style ={ circle ,top color =white , bottom color = processblue!20 ,
               draw,processblue , text=blue , minimum width =1 cm}]
           
           %\node [estilo del nodo] (nombre del nodo) [posicion relativa]/at (posicion,absoluta) {};
           
           \node[state] (A) at (-5, 0) {A};
           \node[state] (B) at (-2.5, 1.5) [label = above: $ 6(I) $] {B};
           \node[state] (D) at (2.5, 1.5) {D};
           \node[state] (F) at (5, 0) {F};
           \node[state] (E) at (2.5, -1.5) {E};
           \node[state] (C) at (-2.5, -1.5) [label = below: $ 7(I) $] {C};
           
           \path (A) edge node[above=0.15] {$8(I)$} (B);
           \path (A) edge (C);
           
           
           \path (B) edge (D);
           
   
           \path (C) edge node[above = 0.15] {$10$} (D);
           \path (C) edge node[below = 0.15] {$5(II)$} (E);
           
   
           \path (D) edge (E);
           \path (D) edge (F);
   
           \path (E) edge (F);
   
   
   
   
           %si queremos un loop
           %\path (A) edge [loop left] node[left] {$1/4$} (A);
           
       \end{tikzpicture}
        \caption{}
   \end{figure}  
   