\begin{figure}[h]

    \begin{tikzpicture}[-latex, auto, node distance =4 cm and 5cm, on grid , semithick, scale = 1.0,
               state/.style ={ circle ,top color =white , bottom color = processblue!20 ,
               draw,processblue , text=blue , minimum width =1 cm}]
           
           %\node [estilo del nodo] (nombre del nodo) [posicion relativa]/at (posicion,absoluta) {};
           
           \node[state] (I) at (-5, 0) {I};
           \node[state] (1) at (-2.5, 2.5) {1};
           \node[state] (2) at (-2.5, -2.5) {2};
           \node[state] (3) at (0, 0) {3};
           \node[state] (4) at (2.5, 2.5) {4};
           \node[state] (F) at (5, 0) {F};

           
           \path (I) edge [bend right = -15] node[above=0.15] {$I(10)$} (1);
           \path (I)[camino] edge [bend right = 15] node[above=0.25] {$II(8)$} (2);
           
           \draw[dashed] (1) --  node[right=0.25] {$0$} (2);
           
           \path (1) edge [bend right = -15] node[above=0.25] {$IV(2)$} (3);
           \path (2)[camino] edge [bend right = 15] node[below=0.35] {$III(12)$} (3);

           \draw[dashed] (3) --  node[right=0.25] {$0$} (4);
           \path (3)[camino] edge [bend right = 15] node[above=0.25] {$VI(20)$} (F);
           \path (4) edge [bend right = -15] node[above=0.25] {$V(15)$} (F);
           
        
   
           %si queremos un loop
           %\path (A) edge [loop left] node[left] {$1/4$} (A);
           
       \end{tikzpicture}
        \caption{En azul se muestra el camino crítico del proyecto.}
   \end{figure}  