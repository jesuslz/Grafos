\begin{figure}[h]

    \begin{tikzpicture}[-latex, auto, node distance =4 cm and 5cm, on grid , semithick, scale = 1.0,
               state/.style ={ circle ,top color =white , bottom color = processblue!20 ,
               draw,processblue , text=blue , minimum width =1 cm}]
           
           %\node [estilo del nodo] (nombre del nodo) [posicion relativa]/at (posicion,absoluta) {};
           
           \node[state] (1) at (-5, 0) [label = left: $(-{,}0)$] {1};
           \node[state] (2) at (-2.5, 2.5) [label = above: $(1{,}0.2)$] {2};
           \node[state] (4) at (2.5, 2.5) [label = above left: $(2{,}1.0)$] [label = above right: $(3{,}1.0)$] {4};
           \node[state] (6) at (5, 0) [label = above: $(4{,}1.35)$] {6};
           \node[state] (5) at (2.5, -2.5) [label = below: $(4{,}1.4)$] {5};
           \node[state] (3) at (-2.5, -2.5) [label = below: $(1{,}0.9)$] {3};
           \node[state] (7) at (6, -1.5) [label = below: $(6{,}1.85)$] {7};

           
           \path (1)[camino] edge node[above=0.15] {$0.2$} (2);
           \path (1) edge node[above=0.15] {$0.9$} (3);

           \path (2) edge node[right=0.15] {$0.6$} (3);
           \path (2)[camino] edge node[above=0.15] {$0.8$} (4);
           
           \path (3) edge node[above=0.15] {$0.1$} (4);
           \path (3) edge node[above=0.15] {$0.3$} (5);

           \path (4) edge node[right=0.15] {$0.4$} (5);
           \path (4)[camino] edge node[above=0.15] {$0.35$} (6);

           \path (6)[camino] edge node[right=0.15] {$0.5$} (7);
           \path (5) edge node[above=0.15] {$0.25$} (7);
        
   
           %si queremos un loop
           %\path (A) edge [loop left] node[left] {$1/4$} (A);
           
       \end{tikzpicture}
        \caption{La figura muestra uno de los dos posibles caminos más económicos.}
   \end{figure}  