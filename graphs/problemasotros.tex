\begin{figure}[h]

 \begin{tikzpicture}[auto, node distance =4 cm and 5cm, on grid , semithick, scale = 1.0,
            state/.style ={ circle ,top color =white , bottom color = processblue!20 ,
            draw,processblue , text=blue , minimum width =1 cm, scale = 1.0}]
        
        %\node [estilo del nodo] (nombre del nodo) [posicion relativa]/at (posicion,absoluta) {};
        
        \node[state] (A) at (-5, 0) {A};
        \node[state] (B) at (-2.5, 3.5) {B};
        \node[state] (C) at (2.5, 3.5) {C};
        \node[state] (D) at (5, 0) {D};
        \node[state] (E) at (2.5, -3.5) {E};
        \node[state] (F) at (-2.5, -3.5) {F};
        
        \path (A) edge [bend right = -25] (B);
        \path (A) edge [bend right = -25] (C);
        \path (A) edge [bend right = -25] (D);
        \path (A) edge [bend right = 25] (E);
        \path (A) edge [bend right = 25] (F);
        
        \path (B) edge [bend right = -25] (C);
        \path (B) edge [bend right = -25] (D);
        \path (B) edge [bend right = 25] (E);
        \path (B) edge [bend right = 25] (F);

        \path (C) edge [bend right = -25] (D);
        \path (C) edge [bend right = -25] (E);
        \path (C) edge [bend right = -25] (F);

        \path (D) edge [bend right = -25] (E);
        \path (D) edge [bend right = -25] (F);

        \path (E) edge [bend right = -25] (F);




        %si queremos un loop
        %\path (A) edge [loop left] node[left] {$1/4$} (A);
        
    \end{tikzpicture}
     \caption{hola muchbaibdfunl}
\end{figure}  
