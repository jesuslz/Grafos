\begin{figure}[h]


    \begin{tikzpicture}[-latex, auto, node distance =4 cm and 5cm, on grid , semithick, scale = 1.0,
            state/.style ={ circle ,top color =white , bottom color = processblue!20 ,
            draw,processblue , text=blue , minimum width =1 cm, scale = 1.0}]
        
            %\node [estilo del nodo] (nombre del nodo) [posicion relativa]/at (posicion,absoluta) {};
            \node[state] (I) at (0, 0) {I};
            \node[state] (n1) at (2.5, 2.5) {1};
            \node[state] (n2) at (2.5, 0.5) {2};
            
            \node[state] (n4) at (4.5, 1.5) {4};
            \node[state] (n5) at (6.5, 1.5) {5};
            \node[state] (n6) at (8.5, 0.5) {6};
            \node[state] (n7) at (9.5, -1.5) {7};
        
    
            \path (I) edge [bend right = -25] node[above = 0.15 cm] {$1 -2$} (n1);
            \path (I) edge node[below = 0.0 cm] {$1 -3$} (n2);
            \path (I) edge [bend right = 65] node[above = 0.15 cm] {$1 -4$} (n4);
            \path (I) edge [bend right = 105] node[above = 0.15 cm] {$1 -5$} (n5);

            \path (n1) edge [bend right = -15] node[above = 0.15 cm] {$2 -4$} (n4);
            \path (n2) edge node[above = 0.15 cm] {$3 -4$} (n4);

            \path (n4) edge [bend right = -15] node[above = 0.15 cm] {$4 -5$} (n5);
            \path (n5) edge [bend right = -15] node[above = 0.15 cm] {$5 -6$} (n6);
            \path (n6) edge [bend right = -15] node[right = 0.15 cm] {$6 -7$} (n7);
    
    
            %si queremos un loop
            %\path (A) edge [loop left] node[left] {$1/4$} (A);
        \end{tikzpicture}
        
        \caption{}
    \end{figure}