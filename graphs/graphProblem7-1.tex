\begin{figure}[h]

 \begin{tikzpicture}[-latex, auto, node distance =4 cm and 5cm, on grid , semithick, scale = 1.0,
            state/.style ={ circle ,top color =white , bottom color = processblue!20 ,
            draw,processblue , text=blue , minimum width =1 cm, scale = 1.0}]
        
        %\node [estilo del nodo] (nombre del nodo) [posicion relativa]/at (posicion,absoluta) {};
        
        \node[state] (n1) at (-5, 0) {1};
        \node[state] (n2) at (-2.5, 2.5) {2};
        \node[state] (n4) at (2.5, 2.5) {4};
        \node[state] (n6) at (5, 0) {6};
        \node[state] (n5) at (2.5, -2.5) {5};
        \node[state] (n3) at (-2.5, -2.5) {3};
        
        \path (n1) edge node[above = 0.15 cm] {$1$} (n2);
        \path (n1) edge node[above = 0.15 cm] {$3$} (n3);

        \path (n2) edge node[above = 0.15 cm] {$3$} (n4);
        \path (n2) edge node[above = 0.15 cm, pos = 0.20] {$2$} (n5);

        \path (n3) edge node[right = 0.15 cm] {$1$} (n2);
        \path (n3) edge node[above = 0.15 cm, pos = 0.20] {$1$} (n4);
        \path (n3) edge node[above = 0.15 cm] {$2$} (n5);

        \path (n5) edge node[left = 0.15 cm] {$1$} (n4);

        \path (n4) edge node[above = 0.15 cm] {$4$} (n6);
        




        %si queremos un loop
        %\path (A) edge [loop left] node[left] {$1/4$} (A);
        
        %otras opciones: 
        %\path (A) 
        %    edge [left] node [blue, pos=0.5, sloped, above] {$0 \rightarrow [x = x.0.0]$} (B)
        %    edge [left] node [cyan, pos=0.8]{$1 \rightarrow [x = x.0.1]$} (C)(B) 
        %    edge [loop above] node [align=center] {$0 \rightarrow$ \\ $[x = x.0]$ }   (B)
        %    edge [bend right,left] node  {$1 \rightarrow [x = x.1]$ }   (C)
        %    edge [] node [red, pos=0.2] {$\$  \rightarrow  [x = x.0.\$]$ } (D);  


    \end{tikzpicture}
     \caption{}
\end{figure}  
